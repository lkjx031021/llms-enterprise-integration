  \begin{tikzpicture}
     \begin{axis}[width=14cm, height=7cm, 
     xmin=370, 
     xmax=730, 
     ymin=,
     ymax=,
     ymajorgrids=true,
    xmajorgrids=true,
    minor tick style={gray!80, thick},
    major tick style={gray, thick},
    minor tick num=1,
    %ytick={0,0.5,1,1.5,...,2},                       % <----- eigene Achsenbeschriftung
    grid style=dashed,                           % <----- dotted oder 
    legend pos= south west,                       % <----- Position der Legende
    legend style={at={(0.5,1.05)},              % <----- Legende außerhalb
	anchor=south,legend columns=-1, draw=none}, % <----- Legende außerhalb
    legend style={draw=none, nodes=right},       % <----- ohne Kasten, links ausgerichtet
    xlabel={\small Wellenlänge [nm]},
    ylabel={\small Absorption [\si{\%}]},
    tick label style={font=\scriptsize},         % <----- Schrift Achsenwerte
    yticklabel style={/pgf/number format/fixed},{% gilt für x und y, auch separat möglich
    /pgf/number format/.cd,
    use comma,% Komma als Dezimaltrenner
    1000 sep = {}% keine Tausendertrennung 
     },
    yticklabel style={
    /pgf/number format/.cd,
    fixed zerofill,  % mit Nullen auffüllen
    precision=1},
     xticklabel style={
    /pgf/number format/.cd,
    fixed zerofill,  % mit Nullen auffüllen
    precision=0},% <----- Nachkommastelle 0 bis 3
    ]
    
    % temp.dat übernimmt Daten von oben
    % data.dat übernimmt Daten aus bestimmter Datei (s. linke Spalte)
    
       %\addplot[mark=o, only marks] table {temp.dat}; 
       
       %\addplot+[mark=, dotted, blue3!20, thick, smooth, mark options={very thick,draw=white}] table {data.dat};

%\shade[left color=blue!20,right color=red!20] (500,1) rectangle (800,200);

\pgfspectra[begin=370,end=730,height=0.3cm,
width=12.42cm % Breite des Spektrums im Diagramm
]
        \addplot+[mark=, red, thick, smooth, mark options={very thick,draw=white}] table {diagram/data_old.dat};
        
       \addplot+[mark=, green2, thick, smooth, mark options={very thick,draw=white}] table {diagram/data_fresh.dat};
       
       %\node[coordinate,pin ={[rotate=90]right:%\scriptsize{\textcolor{gray}{589}
%}}] at (axis cs:587.4,7.9) { };
%\node[coordinate,pin ={[rotate=90]right:%\scriptsize{\textcolor{gray}{765}
%}}] at (axis cs:765.2,5.4) { };

        
        %\addplot[raw gnuplot, smooth, black!30, thick, dotted] gnuplot {plot "data.dat" using ($1):($2/1) smooth sbezier}; %\label{dia:nacl}

       %\addplot+[mark=*, thick, smooth,mark options={very thick,draw=white}] table {data.dat}; %Diese Daten kommen aus der Datei "data.dat"
        
        %\addplot gnuplot [raw gnuplot, smooth, black, thick] {plot 'data.csv' using ($1):($2/1) smooth sbezier}; %\label{dia:nacl}
    

        %\addplot+[mark=*, thick, smooth,mark options={very thick,draw=white}] table [x=Zeit, y=Temperatur, col sep=comma] {data.dat};
        
        %($1/1):($2) Bezierkurven zur Glättung
        %($1):($2/1)
        
        \addlegendentry{\footnotesize{Chlorophyll-alt}}
        \addlegendentry{\footnotesize{Chlorophyll-frisch}}
        %\addlegendentry{\footnotesize{\ch{NH4Cl}}}
        %\addlegendentry{\footnotesize{Stoff B$_2$}} % <- dekommentieren für zweite Kurve
     \end{axis}
   \end{tikzpicture}

%\vspace{0.5cm}
%Referenzspektrum zu Natrium:

%\pgfspectrawrite[na.dat]{Na}
 %\pgfspectra[width=10cm,element = Na, Imin=0.02,begin=400,end=800,axis,axis step=50,axis ticks=4,back=visible40] % oder manuell mit lines={589,589.6}
 % Datenbank NIST, Imin=0.01 alle Linien

%\vspace{0.5cm}
%Mit dem Paket \verb|pgf-spectra| sind viele Spektren und Anpassungen möglich. Das Manual dazu wird verlinkt unter \href{https://ftp.mpi-inf.mpg.de/pub/tex/mirror/ftp.dante.de/pub/tex/graphics/pgf/contrib/pgf-spectra/pgf-spectraManual.pdf}{\texttt{https://ctan.org}}
   %\input{tabelle}