\documentclass{article}

\usepackage{ctex}
\usepackage{amsmath}
\usepackage{minted}
\usepackage{xcolor}

\definecolor{mybgcolor}{RGB}{230, 240, 255}

% 设置全局代码块样式
\setminted{
    bgcolor=mybgcolor,
    frame=lines,
    framesep=2mm,
    rulecolor=blue,
    linenos=true,
    breaklines=true
}

\begin{document}
\title{测试文档}
\author{Yguo. 国越}
\date{\today}
% \date{2025年3月10日}
\maketitle

\tableofcontents

正文内容: $$\sum_{min}^{max}$$

\section{第一个段落。}
\begin{center}
    
第二个段落。第二个段落。第二个段落。第二个段落。第二个段落。第二个段落。第二个段落。第二个段落。第二个段落。第二个段落。第二个段落。第二个段落。第二个段落。第二个段落。第二个段落。第二个段落。第二个段落。第二个段落。第二个段落。第二个段落。第二个段落。第二个段落。第二个段落。第二个段落。第二个段落。第二个段落。第二个段落。第二个段落。第二个段落。第二个段落。第二个段落。第二个段落。第二个段落。第二个段落。第二个段落。第二个段落。第二个段落。

第三个段落.第三个段落.第三个段落.第三个段落.第三个段落.第三个段落.第三个段落.第三个段落.\par
第三个段落.第三个段落.第三个段落.第三个段落.第三个段落.第三个段落.第三个段落.第三个段落.第三个段落.第三个段落.第三个段落.第三个段落.第三个段落.第三个段落.第三个段落.第三个段落.第三个段落.第三个段落.第三个段落.第三个段落.第三个段落.第三个段落.第三个段落.第三个段落.第三个段落.第三个段落.第三个段落.第三个段落.第三个段落.第三个段落.第三个段落.第三个段落.第三个段落.
\end{center}
\subsection{数学公式}
$$
\sigma  = \epsilon \cdot \sqrt{\frac{1}{N} \sum_{i=1}^{N} (x_i - \bar{x})^2}
$$
\begin{minted}{python}
# 假设输入数据
batch_size = 4  # 假设有 4 个样本
channels = 512  # 从 Bottleneck 输出的特征通道
height = 7      # 特征图的高度
width = 7       # 特征图的宽度
spacial_dim = height  # 假设输入特征图是正方形
embed_dim = channels  # 嵌入维度与输入通道数一致
num_heads = 8  # 多头注意力头数
output_dim = 256  # 输出特征维度
    
# 生成模拟输入数据(模拟从 Bottleneck 输出的特征图)
def generate_input_data(batch_size, channels, height, width):
    # 生成随机输入数据
    input_data = torch.randn(batch_size, channels, height, width)  # 随机生成输入数据
    return input_data
input_data = torch.randn(batch_size, channels, height, width) 
\end{minted}

\begin{minted}{bash}
    ./hfd_revised.sh bigcode/starcoderdata --dataset \
    --include "python/*" \
    --include "sql/*" \
    --include "matlab/*" \
    --include "javascript/*" \
    --include "java/*" \
    --include "json/*" \
    --include "c/*" \
    --include "rust/*" \
    --include "go/*" \
    --include "typescript/*" \
    --include "kotlin*" \
    --include "swift/*" \
    --include "julia/*" \
    --include "markdown/*" \
    --include "html/*" \
    --hf_username TsaiTsai0929 \
    --hf_token hf_xxxx \
    --tool aria2c \
    -x 4 -j 5 \
    --local-dir ~/autodl-tmp/dataprocess/data/starcoder
\end{minted}

% \section*{简单的无序列表}
\begin{itemize}
    \item 苹果
    \item 香蕉
    \item 橙子
\end{itemize}

\begin{enumerate}
    \item LORA
    \item PEFT
    \item 3
\end{enumerate}
\begin{description}
    \item[Step1] 低秩适配器  
    \item[Step2]     
\end{description}

\end{document}